  % !TEX root = nsad.tex
  \section{Introduction}
  
  
The design of concurrent data structures is an active research area.  
The first approach towards concurrent data structures uses locks, that ensure synchronization between threads by allowing at most one thread to write (read) a shared resource. 
The problem with lock-based implementations is that they are blocking, i.e., if the thread holding the lock is suspended or crashes before releasing the lock, the entire execution blocks, no other thread being able to acquire a non-released lock. 
Consequently, non-blocking implementations were developed, where the threads that cannot access the shared memory are not blocking, instead retry, report the operation as unsuccessful, etc.  
Typical examples are Treiber's stack~\cite{treiber} or Michael-Scott queue~\cite{mmqueue} (they are implemented using instructions like compare-and-swap).  
However, even for these data structure concurrent accesses are sequentialized at one or two entry points of the data structure, despite its potentially unbounded size. Therefore, the latter data structures suffer also from a performance bottleneck, all accesses to the data structure being done at the top of the stack, respectively the head/tail of the queue, therefore the degree of parallelism is heavily restricted.  

The advantage of these data structures is that they have a sequential specification, easily understood by programmers. More precisely, the standard correctness criteria for concurrent data structures is linearizability, which says that a client cannot distinguish  the concurrent implementation from a sequential (atomic) one. 
The client communicates with the data structure using an API, and the client cannot distinguish  a concurrent implementation of the methods in the API from a sequential one, that we refer to as the specification. We call operation an execution of a method in the API. 
Given their sequential specification, verification methods~\cite{threadmodular, VV1,VV2} and tools~\cite{cave} have been developed, that check a simulation relation between concurrent executions of the (concurrent) implementation and sequential executions in the specification. 
 %cinnrrent sequence of call and returns in a concurrent, respectively a sequential, execution. 
 The simulation is based on the user annotating the concurrent program with the so called linearization points. When a concurrent execution reaches a linearization point,  in the corresponding sequential execution a new operation is added, matching the effect of the concurrent operation whose linearization point was reached.    


More recent non-blocking data structures like, stack-elimination~\cite{} or combined funnels~\cite{}, increase the parallel access to the data structure by implementing a {\em collaboration mechanism}. Roughly, two concurrent operations, e.g., a push and a pop, are allowed to exchange values without accessing the shared stack, using instead a pair of registers. The performance of these data structures emerges in heavily concurrent environments  where hundred of threads try to access the data structure in parallel. The implementation  of stack-elimination uses an auxiliary (elimination) array, where each array cell is a pair of registers that allows two operations to exchange values. The size of the array determines the number of operations that can be done in parallel.  
The exchange mechanism is implemented also as a stand alone data structure, \texttt{java.exchanger}, in the library \texttt{java.conncurent}, being available as a synchronization primitive between two threads. 
All collaboration mechanisms are based on a shared memory implementation of a message passing protocol between the threads that exchange values. 
Therefore these data structures have a concurrent specification~\cite{disc15}, that captures the effect of the collaboration between two threads. 

The verification of data structures that have a concurrent specification has been less studied in the literature. With the exception of~\cite{cav13} all other formal methods techniques tackle the bug finding problem~\cite{EneaEmmie1,EneaEmmie2} or study decidability issues when the number of threads and the size of the data structure are bounded. In~\cite{cav13} the verification method proposed does not addresses directly the verification of cooperation mechanism, instead it reduces it to the verification of a library that has a sequential specification, and can be addressed with existing verification techniques. 



In this paper we study the verification problem for concurrent data structures that have concurrent specifications. We define a trace abstraction of concurrent executions that captures relations between threads, namely the essence of the collaboration between two threads. 
The main ingredients of the abstraction are 1) it uses a bounded number of threads to abstract the behavior of unboundedly many  threads,  2) for each operation it remembers a subset of the events produced, i.e., the call, the return, and writes to the share memory, and 
 3) it is parametrized by an abstract domain that captures the global state of the program projected on boundedtly many threads. 
This is the first approach, as far as we know, that proposes an abstract domain for proving correctness of concurrent data structures that have a concurrent specification.  Our results are preliminary, we have manually tried our abstraction on a few algorithms like, \texttt{java.exchanger}, stack-elimination, and validity, a concurrent object with one method that tries to write a value and returns either the value it wrote if the write was successful, or the value written by a concurrent operation, whose write was successful.   
 
The paper presents in Section~\ref{sec:syntax} the syntax and semantics of the language we consider, Section~\ref{sec:absDom} presents the abstract domain we propose, in Section~\ref{sec:example} we give an example of analysis, and finally related works and conclusions are discussed in Section~\ref{sec:relwork}. 


% 
% Concurrent objects (data structures) are implemented in wide audience libraries such as java.util.concurrent and Intel Threading Building Blocks. The search for new algorithms for concurrent data structures that maximize the degree of parallelism between the operations is an active research area. The algorithms have become increas- ingly complex, which makes proving their linearizability more difficult. Automatic ver- ification techniques are expected not only to confirm the manual correctness proofs provided with the definition of the algorithms, but also to verify the various implemen- tations of these algorithms in different programming languages.